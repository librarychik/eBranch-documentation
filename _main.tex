% Options for packages loaded elsewhere
\PassOptionsToPackage{unicode}{hyperref}
\PassOptionsToPackage{hyphens}{url}
%
\documentclass[
  openany]{book}
\usepackage{lmodern}
\usepackage{amssymb,amsmath}
\usepackage{ifxetex,ifluatex}
\ifnum 0\ifxetex 1\fi\ifluatex 1\fi=0 % if pdftex
  \usepackage[T1]{fontenc}
  \usepackage[utf8]{inputenc}
  \usepackage{textcomp} % provide euro and other symbols
\else % if luatex or xetex
  \usepackage{unicode-math}
  \defaultfontfeatures{Scale=MatchLowercase}
  \defaultfontfeatures[\rmfamily]{Ligatures=TeX,Scale=1}
\fi
% Use upquote if available, for straight quotes in verbatim environments
\IfFileExists{upquote.sty}{\usepackage{upquote}}{}
\IfFileExists{microtype.sty}{% use microtype if available
  \usepackage[]{microtype}
  \UseMicrotypeSet[protrusion]{basicmath} % disable protrusion for tt fonts
}{}
\makeatletter
\@ifundefined{KOMAClassName}{% if non-KOMA class
  \IfFileExists{parskip.sty}{%
    \usepackage{parskip}
  }{% else
    \setlength{\parindent}{0pt}
    \setlength{\parskip}{6pt plus 2pt minus 1pt}}
}{% if KOMA class
  \KOMAoptions{parskip=half}}
\makeatother
\usepackage{xcolor}
\IfFileExists{xurl.sty}{\usepackage{xurl}}{} % add URL line breaks if available
\IfFileExists{bookmark.sty}{\usepackage{bookmark}}{\usepackage{hyperref}}
\hypersetup{
  pdftitle={eBranch documentation},
  pdfauthor={by the eBranch staff},
  hidelinks,
  pdfcreator={LaTeX via pandoc}}
\urlstyle{same} % disable monospaced font for URLs
\usepackage{longtable,booktabs}
% Correct order of tables after \paragraph or \subparagraph
\usepackage{etoolbox}
\makeatletter
\patchcmd\longtable{\par}{\if@noskipsec\mbox{}\fi\par}{}{}
\makeatother
% Allow footnotes in longtable head/foot
\IfFileExists{footnotehyper.sty}{\usepackage{footnotehyper}}{\usepackage{footnote}}
\makesavenoteenv{longtable}
\usepackage{graphicx,grffile}
\makeatletter
\def\maxwidth{\ifdim\Gin@nat@width>\linewidth\linewidth\else\Gin@nat@width\fi}
\def\maxheight{\ifdim\Gin@nat@height>\textheight\textheight\else\Gin@nat@height\fi}
\makeatother
% Scale images if necessary, so that they will not overflow the page
% margins by default, and it is still possible to overwrite the defaults
% using explicit options in \includegraphics[width, height, ...]{}
\setkeys{Gin}{width=\maxwidth,height=\maxheight,keepaspectratio}
% Set default figure placement to htbp
\makeatletter
\def\fps@figure{htbp}
\makeatother
\setlength{\emergencystretch}{3em} % prevent overfull lines
\providecommand{\tightlist}{%
  \setlength{\itemsep}{0pt}\setlength{\parskip}{0pt}}
\setcounter{secnumdepth}{5}

\title{eBranch documentation}
\author{by the eBranch staff}
\date{}

\begin{document}
\maketitle

{
\setcounter{tocdepth}{1}
\tableofcontents
}
\hypertarget{let-the-documentation-begin}{%
\chapter{Let the documentation begin}\label{let-the-documentation-begin}}

TOC

\hypertarget{contents}{%
\chapter{Contents}\label{contents}}

\protect\hyperlink{mission-statement}{Mission Statement:} \textbf{4}

\protect\hyperlink{what-does-the-ebranch-do}{What does the eBranch do?} \textbf{4}

\protect\hyperlink{ebranch-knowledge-base-tools}{eBranch Knowledge Base \& Tools} \textbf{4}

\protect\hyperlink{ebranch-team-approach}{eBranch Team Approach} \textbf{5}

\protect\hyperlink{emergency-communication-responsibilities}{Emergency Communication
Responsibilities} \textbf{5}

\protect\hyperlink{strategies-and-approaches}{Strategies and approaches} \textbf{6}

\begin{quote}
\protect\hyperlink{handy-tips-and-things-to-remember}{Handy tips and things to
remember} 6

\protect\hyperlink{updating-staff-profilesnames}{Updating staff profiles/names} 6
\end{quote}

\protect\hyperlink{routine-ebranch-tasks}{Routine eBranch tasks} \textbf{7}

\begin{quote}
\protect\hyperlink{multisearch-library-search-admin-tool}{MultiSearch / Library Search Admin
Tool} 10

\protect\hyperlink{reserves-in-multisearch}{Reserves in Multisearch} 10
\end{quote}

\protect\hyperlink{web-authors}{Web authors} \textbf{13}

\begin{quote}
\protect\hyperlink{training}{Training} 13

\protect\hyperlink{authoring-and-editing}{Authoring and editing} 13

\protect\hyperlink{best-practices}{Best practices} 15

\protect\hyperlink{types-of-web-authors-how-we-support-them}{Types of Web Authors + how we support
them} 15
\end{quote}

\protect\hyperlink{ebranch-superpowers-and-how-to-wield-them-wisely}{eBranch superpowers and how to wield them
wisely} \textbf{16}

\protect\hyperlink{drupal-adding-an-advancement-donate-now-give-to-the-library-button}{Drupal: Adding an Advancement / Donate now / Give to the Library
button}
\textbf{24}

\protect\hyperlink{password-authentication-on-web-pages}{Password / authentication on web
pages} \textbf{24}

\begin{quote}
\protect\hyperlink{other-odds-and-ends}{Other odds and ends} 26
\end{quote}

\protect\hyperlink{miscellaneous-troubles-that-come-up-and-how-to-shoot-them}{Miscellaneous troubles that come up and how to shoot
them} \textbf{26}

\begin{quote}
\protect\hyperlink{vanishing-attachments}{Vanishing Attachments} 26

\protect\hyperlink{problems-with-godot-pages}{Problems with Godot pages} 27

\protect\hyperlink{problems-with-cufts-pages}{Problems with CUFTS pages} 27

\protect\hyperlink{problems-with-troymillennium-pages}{Problems with Troy/Millennium
pages} 27

\protect\hyperlink{structural-changes-like-taxonomies-and-menus-that-dont-seem-to-take}{Structural changes, like taxonomies and menus, that don't seem to
``take''}
27
\end{quote}

\protect\hyperlink{digital-sign-in-front-door-vestibule-at-bennett-library}{Digital sign in front door vestibule at Bennett
Library}
\textbf{28}

\protect\hyperlink{new-books-list}{New books list} \textbf{28}

\protect\hyperlink{computer-availability-tool}{Computer availability tool} \textbf{29}

\protect\hyperlink{website-migration-2015-leftover-bits-and-bobs}{Website migration 2015 leftover bits and
bobs} \textbf{29}

\begin{quote}
\protect\hyperlink{webform-data-from-old-site}{Webform data from old site:} 29
\end{quote}

\protect\hyperlink{drupal-8-instructions-staff-site}{Drupal 8 instructions (staff site)}
\textbf{29}

\begin{quote}
\protect\hyperlink{location-of-old-and-new-staff-sites}{Location of old and new staff
sites} 29

\protect\hyperlink{access-to-the-staff-site-and-user-account-creation}{Access to the staff site and user account
creation} 29

\protect\hyperlink{roles-and-permissions}{Roles and permissions} 30

\protect\hyperlink{nightly-updates-to-keep-access-current}{Nightly updates to keep access
current} 31

\protect\hyperlink{personal-sfu-computing-ids-that-dont-appear-in-the-staff-sites-drupal-account-list}{Personal SFU computing IDs that don't appear in the staff site's
Drupal account
list}
32

\protect\hyperlink{role-accounts-and-the-white-list}{Role accounts and the white list}
33

\protect\hyperlink{setting-up-access-for-a-role-account}{Setting up access for a role
account} 33

\protect\hyperlink{possible-user-confusion-between-role-accounts-and-web-author-accounts}{Possible user confusion between role accounts and web author
accounts}
33

\protect\hyperlink{creating-a-local-non-cas-account-an-account-where-there-is-no-sfu-computing-id}{Creating a local, non-CAS account (an account where there is no SFU
Computing
ID)}
33

\protect\hyperlink{megamenu}{Megamenu} 33

\protect\hyperlink{content-types}{Content types} 34

\protect\hyperlink{webforms}{Webforms} 34

\protect\hyperlink{incident-reports}{Incident reports} 34

\protect\hyperlink{email-notifications}{Email notifications} 34

\protect\hyperlink{search-index-configuration}{Search index configuration} 35

\protect\hyperlink{search-boxes-on-sets-of-pages}{Search boxes on sets of pages} 35

\protect\hyperlink{staff-site-search-statistics-in-piwik}{Staff site search statistics in
Piwik} 35

\protect\hyperlink{opt-for-drupal-.html-pages-over-attached-files-such-as-.pdf-documents}{Opt for Drupal .html pages over attached files such as .pdf
documents}
35
\end{quote}

\protect\hyperlink{drupal-7-instructions-public-site}{Drupal 7 instructions (public
site)} \textbf{36}

\begin{quote}
\protect\hyperlink{site-location}{Site location} 36

\protect\hyperlink{menus}{Menus} 36

\protect\hyperlink{menu-editing}{Menu editing} 36

\protect\hyperlink{taxonomies}{Taxonomies} 37

\protect\hyperlink{contexts}{Contexts} 37

\protect\hyperlink{search-boxes}{Search boxes} 38

\protect\hyperlink{content-types-1}{Content types} 38

\protect\hyperlink{blogs}{Blogs} 39

\protect\hyperlink{blog-posts}{Blog posts} 39

\protect\hyperlink{blog-taxonomy}{Blog taxonomy} 40

\protect\hyperlink{blog-roll}{Blog roll} 40

\protect\hyperlink{setting-up-a-new-blog}{Setting up a new blog} 40

\protect\hyperlink{contact-us}{Contact us} 41

\protect\hyperlink{home-pages}{Home pages} 41

\protect\hyperlink{news-and-events-and-faqs}{News and events and FAQs} 41

\protect\hyperlink{rotating-images}{Rotating images} 41

\protect\hyperlink{forms-webforms}{Forms/ webforms} 42

\protect\hyperlink{attached-file-fields-in-forms}{Attached file fields in forms} 42

\protect\hyperlink{site-wide-webform-settings}{Site-wide webform settings} 42

\protect\hyperlink{image-gallery}{Image gallery} 42

\protect\hyperlink{faqs}{FAQs} 43
\end{quote}

\protect\hyperlink{systems-wiki}{Systems wiki} \textbf{43}

\textbf{\protect\hyperlink{personnel-and-staffing-info}{Personnel and staffing info} 43}

\begin{quote}
\protect\hyperlink{vacation-scheduling}{Vacation scheduling} 43
\end{quote}

\hypertarget{details}{%
\chapter{Details}\label{details}}

\hypertarget{mission-statement}{%
\section{Mission Statement:}\label{mission-statement}}

\begin{enumerate}
\def\labelenumi{\Alph{enumi}.}
\item
  We believe Library resources should be as \textbf{easy to use} as possible. We value plain language, simple and intuitive layouts, and \textbf{accessible} tools and resources in every sense of the word.
\item
  Our focus is \textbf{end users}: students, faculty, staff, SFU administrators, and the general public. In practice this includes \textbf{supporting web authors} as well.
\item
  The eBranch is \textbf{an important link between public services and
  \textgreater{} technical processes}. Some of the partners we work with most
  \textgreater{} closely:

  \begin{itemize}
  \item
    Public services: Liaison librarians, SLC, RC, L \& I, Loans,
    \textgreater{} Belzberg and Fraser locations.
  \item
    Management office, including Graphics.
  \item
    Technical services: Systems, and Collections\textbf{\emph{.}}
  \end{itemize}
\end{enumerate}

\hypertarget{what-does-the-ebranch-do}{%
\chapter{What does the eBranch do?}\label{what-does-the-ebranch-do}}

The eBranch has overall responsibility for user experience for the
Library's online presence:

\begin{itemize}
\item
  Library Search
\item
  Hours tool
\item
  Blogs (Feedback and Staff blogs, plus blogs by individual
  \textgreater{} librarians)
\item
  Workshops
\item
  Other miscellaneous pages, such as: Librarian and other Library
  \textgreater{} Position Openings
  \textgreater{} (\href{http://www.lib.sfu.ca/about/positions}{{http://www.lib.sfu.ca/about/positions}})
\end{itemize}

The eBranch provides training and support to the Library's authors:

\begin{itemize}
\item
  Evaluating and recommending improvements (at a high-level and to
  \textgreater{} individual pages) to increase usability and function, and
  \textgreater{} implementing the recommended changes.
\item
  Assisting web authors to create and maintain webpages, online
  \textgreater{} resources, and posts to the Library webpages and social media.
\item
  Writing, editing, formatting, and posting items to the Library's
  \textgreater{} website.
\item
  Creating and supporting web forms.
\end{itemize}

The eBranch is the link between public services and Library Systems:

\begin{itemize}
\item
  Communicating internally to systems if the public cannot access our
  \textgreater{} services
\item
  Communicating externally thru News and Events and Notices on the
  \textgreater{} Home Page
\end{itemize}

\hypertarget{ebranch-knowledge-base-tools}{%
\chapter{eBranch Knowledge Base \& Tools}\label{ebranch-knowledge-base-tools}}

\begin{itemize}
\item
  Software (e.g.~Drupal)
\item
  Usability and standards
\item
  Writing for the web
\item
  Online accessibility
\item
  Web Authors Guidelines:
  \textgreater{} \href{http://staff.lib.sfu.ca/divisions/ebranch/publishing/writing}{{http://staff.lib.sfu.ca/divisions/ebranch/publishing/writing}}
\item
  Using Drupal:
  \textgreater{} \href{http://staff.lib.sfu.ca/divisions/ebranch/publishing/drupal}{{http://staff.lib.sfu.ca/divisions/ebranch/publishing/drupal}}
\item
  eBranch Tools:
  \textgreater{} \href{http://staff.lib.sfu.ca/divisions/ebranch/tools}{{http://staff.lib.sfu.ca/divisions/ebranch/tools}}

  \begin{itemize}
  \tightlist
  \item
    must be logged in as an admin to access the content (note that
    \textgreater{} you will need the public or staff admin login depending on the
    \textgreater{} location)
  \end{itemize}
\item
  x: KeePass for passwords and also URLs for web admin functions
\end{itemize}

\hypertarget{ebranch-team-approach}{%
\chapter{eBranch Team Approach}\label{ebranch-team-approach}}

\begin{itemize}
\item
  Our team includes web developers, librarians, UX professionals and a
  \textgreater{} manager.
\item
  Information sharing:

  \begin{itemize}
  \item
    Informal and frequent (as-needed) communication
  \item
    staff web pages
  \item
    Google Docs (be sure to share your Google ID)
  \item
    X drive
  \item
    P Keep: eBranch (less so)
  \item
    Calendars: eBranch team members share calendars and keep them up
    \textgreater{} to date
  \end{itemize}
\item
  Working from home (or off-site): Let manager know and mark in
  \textgreater{} calendar
\end{itemize}

\hypertarget{emergency-communication-responsibilities}{%
\chapter{Emergency Communication Responsibilities}\label{emergency-communication-responsibilities}}

\begin{itemize}
\tightlist
\item
  \textbf{Communicating in emergency situations--- When disaster strikes or
  \textgreater{} snow procedures}
\end{itemize}

\begin{enumerate}
\def\labelenumi{\arabic{enumi}.}
\item
  In the event of an emergency an ADL or the UL will update the public
  \textgreater{} site's home page with an emergency message using the Emergency
  \textgreater{} page content type.
\item
  \textbf{The eBranch is responsible for the} \textbf{hours tool reflecting our
  \textgreater{} hours during an emergency or snow closure. }
\item
  \textbf{Wait to hear from the UL or an ADL before making changes to the
  \textgreater{} hours tool} or, if you think it a good idea, adding in an
  \textgreater{} emergency message. Don't make changes based on what SFU University
  \textgreater{} Communications states as their messaging can be ambiguous and we
  \textgreater{} don't wish to erroneously declare a library closed or its hours
  \textgreater{} truncated.
\item
  Whoever sees a communication about changed campus hours first should
  \textgreater{} make the appropriate changes to the hours tool and let all of us
  \textgreater{} eBranch staff know that you're doing it so only one person is
  \textgreater{} updating hours at one time.
\end{enumerate}

\begin{itemize}
\tightlist
\item
  \textbf{Responding to crises---e.g.~a bug happens}. We are here to test,
  \textgreater{} test report, and communicate both ways
\end{itemize}

\begin{enumerate}
\def\labelenumi{\arabic{enumi}.}
\item
  Try to replicate the problem
\item
  Acknowledge the issue -- let the person know you are going to work
  \textgreater{} on it. Ask them to step away so that we can work on it (i.e.~they
  \textgreater{} must stop editing the problem page)
\item
  Once you have a decent test or understand the problem, report it,
  \textgreater{} usually to Todd (or Kurt if it's a server issue). When in doubt,
  \textgreater{} try lib-sys. It is better to report than not to report.
\item
  Status updates to the person who initially reported the problem.
\item
  Repeat as necessary.
\end{enumerate}

\begin{itemize}
\tightlist
\item
  \textbf{Communicating in emergency situations---e.g.~server down}
\end{itemize}

\begin{enumerate}
\def\labelenumi{\arabic{enumi}.}
\item
  Wait a couple of minutes. Often the issue is temporary and will fix
  \textgreater{} itself.
\item
  E-mail lib-sys to let them know about the problem. Mark as URGENT.
\end{enumerate}

\begin{itemize}
\tightlist
\item
  \textbf{Communicating in (semi) emergency situations---Library Search,
  \textgreater{} Catalogue not working}
\end{itemize}

\begin{enumerate}
\def\labelenumi{\arabic{enumi}.}
\item
  Create a News \& Events item describing the problem, e.g.~Library
  \textgreater{} Search is experiencing difficulties\ldots. \[temporary solution\]\ldots{}
  \textgreater{} We are working on the problem." Then select ``High Priority'' so
  \textgreater{} that an Attention/warning icon will appear, and the item will stay
  \textgreater{} at the top of the display. (Sample wording: Library Search
  \textgreater{} experiencing problems: We are currently having problems with some
  \textgreater{} functionality in Library Search. We are working to resolve the
  \textgreater{} issue. If you receive an error message (such as Application Error)
  \textgreater{} when searching for books, journal articles, or media, you can
  \textgreater{} still use the \href{https://sfu-primo.hosted.exlibrisgroup.com/primo-explore/search?vid=SFUL\&sortby=rank}{{Library
  \textgreater{} Catalogue}}
  \textgreater{} to find your resources.
\item
  If a Multisearch/Library Search problem, go to the Multisearch Admin
  \textgreater{} Tool (\url{http://search.lib.sfu.ca:8001/})) and select Notices; either
  \textgreater{} edit an existing notice or add a new one to have it appear
  \textgreater{} (prominently!) on the Library Search results pages.
\end{enumerate}

\end{document}
